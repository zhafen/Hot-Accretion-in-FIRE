\begin{document}

\textbf{Hot accretion or not?}
It takes effort to convince the audience it's hot accretion with the current plots.

\textbf{How many particles are included?}
Some audience members may get \textit{seriously} hung up on how many particles are included in each plot.
I.e., what happens to particles that never heat above 1e5K or cool below it?

\textbf{Does the 1e5K cut work for m11 halos?}
Audience members worry about the 1e5K cut relative to halo virial temps.

\textbf{Is $t_{\rm acc}$ better than $t_{T=10^5\,{\rm K}}$?}
Should we replace the axes for most plots?
Especially for m11d in the appendix?

\textbf{Are there missing heating terms?}
Cooling vs heating plot for cosmic rays doesn't match temperature.
May be fixed by using total radiative energy (including UVB), not just cooling.

\textbf{What is the point of Fig 9 (Mvir vs thin disk)?}
It doesn't clearly show that faligned is better than Mvir at predicting thin disk.

\textbf{What are the conditions required to produce low-mass subsonic halos?}
Emphasize baryon depletion.
I think we might do this already.

\textbf{What does the prevalence figure mean?}
Some audience members don't see the y-axis value as indicating cooling flow accretion.

\textbf{When does alignment happen? ISM or CGM?}
This is in the text, but requires a two-step connection:
cooling happens at galaxy edge,
and angular momentum happens before then.
Changing the timescale to relative to accretion might help.
Currently it requires some convincing.

\textbf{What's the relationship between accretion, cooling, and aligning?}
I use these interchangeably because I'm comfortable, but the audience will get confused.
Pick one and stick with it.
Maybe make a figure that plots each against the other.

\textbf{What plot do you have that shows rotating cooling flows are a necessary condition for thin disks?}
The prevalence figures didn't convince the audience.
They show a correlation, but not necessarily causation.

\textbf{Can alignment happen in the ISM?}
Some audience members think it can, if they think of the turbulence crossing timescale as the mixing time.
We don't show it can't, so we need to be careful about this.

\textbf{How does the AM distribution of the CGM compare to the AM distribution of the galaxy?}
Relevant to previous results and interpreting angular momentum exchange in the CGM.

\textbf{Should we emphasize angular momentum conservation?}
Can be a distraction if people think AM is being radiated away.

\end{document}