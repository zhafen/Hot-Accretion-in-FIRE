\begin{document}

\section{General Questions}

\textbf{Hot accretion or not?}
It takes effort to convince the audience it's hot accretion with the current plots.

\textbf{How many particles are included?}
Some audience members may get \textit{seriously} hung up on how many particles are included in each plot.
I.e., what happens to particles that never heat above 1e5K or cool below it?

\textbf{Does the 1e5K cut work for m11 halos?}
Audience members worry about the 1e5K cut relative to halo virial temps.

\textbf{Is $t_{\rm acc}$ better than $t_{T=10^5\,{\rm K}}$?}
Should we replace the axes for most plots?
Especially for m11d in the appendix?

\textbf{Are there missing heating terms?}
Cooling vs heating plot for cosmic rays doesn't match temperature.
May be fixed by using total radiative energy (including UVB), not just cooling.

\textbf{What is the point of Fig 9 (Mvir vs thin disk)?}
It doesn't clearly show that faligned is better than Mvir at predicting thin disk.

\textbf{What are the conditions required to produce low-mass subsonic halos?}
Emphasize baryon depletion.
I think we might do this already.

\textbf{What does the prevalence figure mean?}
Some audience members don't see the y-axis value as indicating cooling flow accretion.

\textbf{When does alignment happen? ISM or CGM?}
This is in the text, but requires a two-step connection:
cooling happens at galaxy edge,
and angular momentum happens before then.
Changing the timescale to relative to accretion might help.
Currently it requires some convincing.

\textbf{What's the relationship between accretion, cooling, and aligning?}
I use these interchangeably because I'm comfortable, but the audience will get confused.
Pick one and stick with it.
Maybe make a figure that plots each against the other.

\textbf{What plot do you have that shows rotating cooling flows are a necessary condition for thin disks?}
The prevalence figures didn't convince the audience.
They show a correlation, but not necessarily causation.

\textbf{Can alignment happen in the ISM?}
Some audience members think it can, if they think of the turbulence crossing timescale as the mixing time.
We don't show it can't, so we need to be careful about this.

\textbf{How does the AM distribution of the CGM compare to the AM distribution of the galaxy?}
Relevant to previous results and interpreting angular momentum exchange in the CGM.

\textbf{Should we emphasize angular momentum conservation?}
Can be a distraction if people think AM is being radiated away.

\section{Person-by-person Feedback}

\textbf{C-A} - compiled
coauthor

\textbf{Drummond} - compiled
coauthor?

\textbf{Daniel} - compiled
coauthor?

\textbf{Andrew} - compiled
coauthor

\textbf{Mike B-K} - compiled
coauthor

\textbf{Jorge} - compiled
coauthor

\textbf{Tjitske} - compiled
coauthor?

\textbf{Kareem} - compiled
coauthor

\textbf{TK} - compiled
coauthor?

\textbf{Cameron T} - compiled
coauthor?

\textbf{Dusan} - will not hear from
coauthor?

\textbf{Alex}
coauthor

\textbf{Eliot}
coauthor?

\textbf{Cameron H}
coauthor?

\textbf{Viraj}
coauthor?

\section{Changes to Implement}

\subsection{Figures}

% Change all labels to $t_{10^{4.5}\,{\rm K}}$.

% Remove all metal diffusion markers.
% Say in the text there are no significant differences.

Fig.~\ref{f: overview}:
% Use consistent $\tcon$ label.
Add mean temperature to each of the small panels.
% Add aligned fraction to each of the small panels?

Fig.~\ref{f: before and after A}:
% Try putting z and r in the same plot in.
% Add sound speed line to velocity panel.
% Why not add z/R to this plot?
Change $R$ to $r$.

Fig.~\ref{f: theta vs t}:
% Add $\Delta f_{\aligned}$ values to each panel.

Fig.~\ref{f: prevalence}:
% Left panel is okay, but distracting, and isn't even that good for explaining.
% Try changing y-axis to hot accretion fraction.
Add $f_{\rm aligned}$, now that it's in the previous figure.

Fig.~\ref{f: before and after B}:
% Change median jz/j to Lz/L.
% Change $R v_c$ to lowercase.
% Remove minimum star formation density line.
% Add median accretion time vertical line.

Fig.~\ref{f: prevalence - angular momentum}:
% Remove circle around m12m in angular momentum prevalence figure.
Change units to Gyr.
Try just using the spin-aligned fraction at time of accretion.

Fig.~\ref{f: coherence}:
% That the gas aligns prior to accretion is not clear enough.

Fig.~\ref{f: before and after m12i cr}:
Move to main text?
Add effective sound speed line.
Use same y range.

Fig.~\ref{f: prevalence vs galaxy properties}:
Try to homogenize font sizes.
Play with scaling size by Mvir and dropping left panel.

Fig.~\ref{f: sample validation -- spatial}:
% Add <90\% line to plot.
% Move temperature to bottom panel.

\subsection{Text}

How much have the halos in the sample been selected on having a relative quiet recent history (i.e. no major mergers)? Has there been any selection on appearance or history of the halos you use? I think that would be good to make more explicitly clear in the paper. 

Don't rely on the readers remembering what $\Delta f_{\rm aligned}$ is, and the definition of it seems somewhat circular.

Change $\tcon$ to $t_{10^{4.5}\,{\rm K}}$.

Add an introduction sentence to the abstract about the importance of studying disk formation.
The start of the abstract now feels a bit abrupt to me and requiring some understanding of the major questions from the audience.

Add James' idea, that rotational support provides time to cohere.

Notate how many particles are tracked.

Krumholz+Burkhart have work on large $\sigma$ driven by gravitational energy from accretion which may play a role in explaining high $\sigma$ at high $z$.
Look into this.

Lots of confusion about ``deceleration''.
Be very explicit.

Lots of confusion about ``stalling''.
Be very explicit.

Start with the facts for "the discrepancy with low-mass disk galaxies".

\S\ref{s: methods -- analysis}:
Add a few more words to motivate the temperature cut of $\tcon$.
say more how you selected this gas - is this the progenitor of all star and gas particles in the disk at z = 0 that accreted during this time interval?
also, you are computing $\tcon$ separately for each particle? so in that figure are you binning the particles into those with $\tcon$ near that value?

\S\ref{s: mechanics -- energy balance}:
Make it clear the low-mass galaxies \textbf{don't} have qualitatively similar energy balance.

\S\ref{s: discussion -- observations}:
Our comparison to observations section needs work.
Remove X-ray observations comparison to observations section.

\S\ref{s: appendix-low mass}:
Join appendices and add text.

\section{Changes Implemented}

Re-added more references to Cameron's work.

Define $t^{(s)}_{\rm cool}$.

\end{document}