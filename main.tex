%%%%%%%%%%%%%%%%%%%%%%%%%%%%%%%%%%%%%%%%%%%%%%%%%%
% Basic setup. Most papers should leave these options alone.
\documentclass[fleqn,usenatbib]{mnras}

% MNRAS is set in Times font. If you don't have this installed (most LaTeX
% installations will be fine) or prefer the old Computer Modern fonts, comment
% out the following line
\usepackage{newtxtext,newtxmath}
% Depending on your LaTeX fonts installation, you might get better results with one of these:
%\usepackage{mathptmx}
%\usepackage{txfonts}

% Use vector fonts, so it zooms properly in on-screen viewing software
% Don't change these lines unless you know what you are doing
\usepackage[T1]{fontenc}

% Allow "Thomas van Noord" and "Simon de Laguarde" and alike to be sorted by "N" and "L" etc. in the bibliography.
% Write the name in the bibliography as "\VAN{Noord}{Van}{van} Noord, Thomas"
\DeclareRobustCommand{\VAN}[3]{#2}
\let\VANthebibliography\thebibliography
\def\thebibliography{\DeclareRobustCommand{\VAN}[3]{##3}\VANthebibliography}


%%%%% AUTHORS - PLACE YOUR OWN PACKAGES HERE %%%%%

% Only include extra packages if you really need them. Common packages are:
\usepackage{graphicx}	% Including figure files
\usepackage{amsmath}	% Advanced maths commands
\usepackage{amssymb}	% Extra maths symbols

%%%%%%%%%%%%%%%%%%%%%%%%%%%%%%%%%%%%%%%%%%%%%%%%%%

%%%%% AUTHORS - PLACE YOUR OWN COMMANDS HERE %%%%%

% Please keep new commands to a minimum, and use \newcommand not \def to avoid
% overwriting existing commands. Example:
%\newcommand{\pcm}{\,cm$^{-2}$}	% per cm-squared

%%%%%%%%%%%%%%%%%%%%%%%%%%%%%%%%%%%%%%%%%%%%%%%%%%

%%%%%%%%%%%%%%%%%%% TITLE PAGE %%%%%%%%%%%%%%%%%%%

% Title of the paper, and the short title which is used in the headers.
% Keep the title short and informative.
\title[Hot Accretion in FIRE]{Galactic Discs Fed by Hot Accretion in FIRE}

% The list of authors, and the short list which is used in the headers.
% If you need two or more lines of authors, add an extra line using \newauthor
\author[\ldots]{
\ldots,$^{1}$\thanks{E-mail: mn@ras.org.uk (KTS)}
\\
% List of institutions
$^1$ \ldots
}

% These dates will be filled out by the publisher
\date{Accepted XXX. Received YYY; in original form ZZZ}

% Enter the current year, for the copyright statements etc.
\pubyear{2020}

% Don't change these lines

\newcommand{\Rcool}{R_{T=10^5\,{\rm K}}}
\newcommand{\tcon}{t_{T=10^5\,{\rm K}}}
\newcommand{\Mdot}{\dot{M}}
\newcommand{\Rcirc}{R_{\rm circ}} %need better name as R_cool means something else

\begin{document}
\label{firstpage}
\pagerange{\pageref{firstpage}--\pageref{lastpage}}
\maketitle

% Abstract of the paper
\begin{abstract}

We study how gas accretes onto $\sim L^\star$ star-forming galaxies at redshift $z\sim0$ using the FIRE cosmological simulations. We evaluate the relative importance of three modes of accretion: gas that never heats to the virial temperature (`cold accretion'), hot gas that condenses into cool clouds in the halo which then accretes onto the galaxy (``condensation''), and gas that remains hot down to the galaxy scale at which point it circularizes and cools (`cooling flow'). 
We demonstrate that across our sample of 4 MW-mass halos the primary mode of gas accretion is `cooling flow', with an inflow rate $\sim xx\times$ that of other accretion modes. Cooling gas changes from an irregular morphology without a preferred direction to strong alignment with the galactic plane within $\sim 100$ Myr of cooling.
Accretion via condensation is only associated with satellites (?). Cold accretion does not occur in significant quantities in our sample. 
% Using particle tracking we demonstrate that gas remains hot until it cools when it circularizes at $\lesssim 0.1 R_{\rm vir}$.
% \textit{Gas cools after circularizing because it is no longer undergoing compressional heating that offsets cooling.}
% \textbf{Check the numbers.}
\end{abstract}

% Select between one and six entries from the list of approved keywords.
% Don't make up new ones.
\begin{keywords}
keyword1 -- keyword2 -- keyword3
\end{keywords}

%%%%%%%%%%%%%%%%%%%%%%%%%%%%%%%%%%%%%%%%%%%%%%%%%%

%%%%%%%%%%%%%%%%% BODY OF PAPER %%%%%%%%%%%%%%%%%%

\subsection{ KEY FOR COAUTHORS}
\textbf{Bold: Notes for things to implement.} \\
\textit{Italics: Rough text, needs polishing.} \\
Normal: Normal text, polished enough to be included in a draft.

\section{Introduction}

% Accretion in cosmological simulations
\textbf{Ho,Martin+2019}

% Modes of hot accretion
\textit{
Two modes of hot accretion:
\begin{itemize}
    \item condensation (Maller \& Bullock 2004; Kaufmann, Bullock+09; McCourt+12; Voit+17)
    \item classic CF (Cowie+80; Paper I)
\end{itemize}
}

% Observational signatures
\textbf{
Kregel+02; van der Kruit+07;  Martín-Navarro+12; Comeron+12 all discuss the observed ``truncation radius'' of disc galaxies roughly at $\sim$4 effective radii.
Our $\Rcool$ may explain this truncation.
}

% Related work
\textbf{Oppenheimer+2018  show increased angular momentum in the inner halo, consistent with our results.
Huscher+2020 also have relevant ang momentum work.}

\section{Methods}

% Simulation sample
\textit{
We use four simulations:
m12i\_md\_7100, m12b\_md\_7100, m12f\_core\_7100, m12m\_core\_7100
}
\textbf{Also do massive galaxies?}
\textbf{Do more metal diffusion simulations (the non-md sims have unexpected cooling at large radii): at least 4 metal diffusion simulations would be good.}

% How we select the particles
\textit{
For a given galaxy we select all particles that are in the central galaxy at $z=0$ and in the CGM 1 Gyr prior.
The galaxy is defined as all gas and stars inside $R_{\rm gal} = 0.1 R_{\rm vir}$, with an additional density cut of $n_{\rm H} = 0.13$ cm$^{-3}$ for gas.
The CGM is defined as all gas inside $0.1 -1 R_{\rm vir}$.
For each selected particle we retrieve the full history of the particle (including temperature, density, metallicity) throughout the simulation.
}

\textbf{The number of IDs that are accreted for m12i is 18554.
However, after removing duplicates, the number of IDs that are tracked is 14703.
Either $\sim20\%$ of the particles split, or this is a symptom of a bug.
Find out which.}

% Definition of t1e5 and calculating it
\textit{
$\tcon$ is the latest time among our particles at which the particle transitions from $T > 10^5$ K to $T< 10^5$ K.
(When selecting values in the simulation corresponding to $\tcon$ we use the last snapshot an accreted particle had $T > 10^5$ K, i.e. the snapshot immediately before the transition.)
Note that we do not account for gas particles being heated to $T > 10^5$ K while still in the ISM.
Because our sample of tracked particles focuses on recently accreted particles this is expected to be a small population that will not contaminate our analysis.
This is confirmed after-the-fact in Figure~\ref{f: theta vs R}, where gas is not preferentially oriented in a disky ISM prior to $\tcon$.
}
\textbf{We could select out ISM-heated particles, if necessary, but as noted above Figure~\ref{f: theta vs R} shows that this is not an issue.}

\begin{equation}
    \Rcool \equiv R(t_{T=10^5\,{\rm K}})
\end{equation}


\textit{the circularization radius $\Rcirc$ is defined via}
\begin{equation}
    j = v_{\rm c}(\Rcirc)\Rcirc
\end{equation}
\textit{where $j$ is the specific angular momentum and $v_{\rm c}$ is the circular velocity.}

\textbf{When calculating the cooling function I just used Y=0.2485. That should be fine, right?}

\section{Results}

\begin{figure}
    \centering
    \includegraphics{Mdot_m12i.pdf}
    \caption{
    Mass inflow in spherical shells as a function of radius for $T < 10^5$ K gas (blue) and $T>10^5$ K (red) gas.
    Most inflowing gas at $r\gtrsim 20 $ kpc is hot.
    \textbf{Add images of face-on $z=0$ slice with temperature and velocity arrows.}
    \textbf{Add SFR$(>r)$}
    % \textbf{Add cuts on density/temperature to avoid ISM gas causing a wide dispersion at low-r.}
    % \textbf{Change to linear space to emphasize halo. Cut out accretion shock.}
    }
    \label{f:Mdot}
\end{figure}

\begin{figure}
    \centering
    \includegraphics[width=\columnwidth]{figures/tracks_m12i_md.pdf}
    \caption{
    \textbf{Main panel:} 10 T vs.\ R and K vs.\ R tracks for an $L^\star$ halo at $z=0$ (\texttt{m12i\_md}), with color indicating time minus accretion time.
    We only show gas that has never been ejected from the central galaxy.
    Gas remains hot until cooling at $\lesssim 0.1 R_{\rm vir}$.
    \textbf{Top panel:} Distribution of $\Rcool$, the radius at which gas particles last transitioned from $T > 10^5$ K to $T < 10^5$ K prior to accreting, i.e. the radius at which they cooled prior to accretion.
    \textbf{Add some way of visually marking the gas that heats up again. Maybe an outline?}
    }
    \label{f: T vs R}
\end{figure}

\begin{figure}
    \centering
    \includegraphics[width=\columnwidth]{figures/tracks_m12b_md.pdf}
    \caption{
    Same as Figure~\ref{f: T vs R}, but for \texttt{m12b\_md}.
    }
    \label{f: T vs R m12b_md}
\end{figure}

\begin{figure}
    \centering
    \includegraphics[width=\columnwidth]{figures/tracks_m11d_md.pdf}
    \caption{
    Same as Figure~\ref{f: T vs R}, but for \texttt{m11d\_md}.
    }
    \label{f: T vs R m12b_md}
\end{figure}

% \begin{figure}
% \centering
% \includegraphics[width=\columnwidth]{figures/tracks_m12i_md.pdf}
% \end{figure}

\begin{figure*}
\includegraphics[width=\columnwidth]{rad_vs_compress.png}
\includegraphics[width=\columnwidth]{rad_vs_compress_zoom.png}
\caption{
Compressional heating rate vs cooling rate of accreted particles as a function of time relative to $t(T=10^5\,{\rm K})$.
Thick lines are medians while thin lines are 16 and 84 percentiles.
Right panels focuses on $-60 - 100$ Myr.
This figure shows that gas cools when the cooling is no longer offset by compressive heating.
\textbf{
Add a vertical line in the left panel at t-t1e5=0.
}
\textbf{
Make labels bigger, and space between figures smaller.
}
\textbf{
Change image type to PDF so it never loses resolution.
}
\textbf{
Consider adding an individual track or a few on top (will definitely need to make the lines into shaded regions if we do this).
}
}
\end{figure*}

\begin{figure}
    \centering
    \includegraphics[width=\columnwidth]{figures/theta_vs_t.pdf}
    \caption{
    Angular distribution of accreting gas over 250 Myr before/after cooling.
    Prior to cooling the gas is distributed without a strong preferential direction (as shown by the consistency with the horizontal line representing a spherical PDF).
    After cooling the gas is primarily found in the disk at $\cos\ \theta = 0$.
    }
    \label{f: theta vs R}
\end{figure}

\begin{figure}
    \centering
    % \includegraphics{}
    \caption{
    \textbf{Spare figure, but maybe
    cooling emission in X-ray, optical and UV lines vs.\ radius, only from tracked particles
    }
    }
    \label{f:emission}
\end{figure}

% Accretion tracks figure description
\textit{
The main panels of Figure~\ref{f: T vs R} show the paths in temperature-radius and entropy-radius space taken by accreting particles.
}

% Distribution of r1e5
\textit{
As an explicit example of how far our distribution is from a condensation model, isotropic cooling occurring randomly within $R_{\rm vir}$ would cool in a wide distribution with a median at $0.8 R_{\rm vir}$, while for our accreted gas the median $\Rcool \approx 0.06 R_{\rm vir}$.
}
\textit{We added all particles with $\Rcool>100$ to $R=100$ bin, so we don't ignore them.}

\section{Discussion}

\subsection{AGN feedback}

\textit{
\begin{itemize}
    \item Results expected to be valid after disc forms (Paper III) and before AGN feedback kicks in (ref.~Byrne+)
    \item Results are baseline for detecting effects of feedback in observations / simulations
\end{itemize}
}

\subsection{Contrast with Condensation/Precipitation}

\textbf{Discuss resolution effects and Balbus\&Soker}

\section{Conclusions}

\textbf{TBD.}

\section*{Acknowledgements}

\textbf{TBD.}


%%%%%%%%%%%%%%%%%%%%%%%%%%%%%%%%%%%%%%%%%%%%%%%%%%

%%%%%%%%%%%%%%%%%%%% REFERENCES %%%%%%%%%%%%%%%%%%

% The best way to enter references is to use BibTeX:

\bibliographystyle{mnras}
\bibliography{example} % if your bibtex file is called example.bib

% Alternatively you could enter them by hand, like this:
% This method is tedious and prone to error if you have lots of references
%\begin{thebibliography}{99}
%\bibitem[\protect\citeauthoryear{Author}{2012}]{Author2012}
%Author A.~N., 2013, Journal of Improbable Astronomy, 1, 1
%\bibitem[\protect\citeauthoryear{Others}{2013}]{Others2013}
%Others S., 2012, Journal of Interesting Stuff, 17, 198
%\end{thebibliography}

%%%%%%%%%%%%%%%%%%%%%%%%%%%%%%%%%%%%%%%%%%%%%%%%%%

%%%%%%%%%%%%%%%%% APPENDICES %%%%%%%%%%%%%%%%%%%%%

\appendix

\section{Angular Momentum of Accreting Material}

% \begin{figure}
%     \centering
%     \includegraphics[width=\columnwidth]{figures/j_vs_rcondense.png}
%     \caption{
%     Distribution of $\Rcool$ vs j($\Rcool$) for four FIRE-2 halos with $L(z=0) \sim L^\star$.
% Thick (thin) contours enclose values for 50\% (90\%) of the accreted gas particles.
% The angular momentum as a function of radius for all gas in \texttt{m12i} at $z=0$ is displayed as a dashed line (the median) and shaded regions (5th-95th percentiles).
% \textbf{
% Maybe delete this figure later, because it's only relevant for simulations that have a wide distribution of $\Rcool$, which are only the artificially wide non-md runs.
% }
% \textbf{Is the 100 kpc-cooling gas related to satellite galaxies?}
% \textbf{Try changing shaded region to only hot gas instead of all gas.}
% \textbf{Try histogram of r/(j/vc) instead, to demonstrate that that decreases the spread.}
% \textit{
% In all halos the distributions are consistent with $j_{\rm c} = v_{\rm c} r$, i.e. gas cools once circularized.
% This demonstrates that the variable angular momentum of incoming gas drives the width in the $\Rcool$ distribution.
% }
%     }
%     \label{f: jcool vs Rcool}
% \end{figure}

\section{Metal diffusion vs non-metal diffusion}

\textbf{Plot of distribution $\Rcool$ for m12i and m12i\_md overlapping. Show cooling only happens at large radii in metal diffusion sims.}

%%%%%%%%%%%%%%%%%%%%%%%%%%%%%%%%%%%%%%%%%%%%%%%%%%


% Don't change these lines
\bsp	% typesetting comment
\label{lastpage}
\end{document}

% End of mnras_template.tex